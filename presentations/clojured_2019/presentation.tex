% Author and Copyright: Ruediger Gad
% This work is licensed under a Creative Commons Attribution-ShareAlike 3.0 Unported License.
% https://creativecommons.org/licenses/by-sa/3.0/
\documentclass[aspectratio=169]{beamer}
%[handout]
\usepackage[english]{babel}
\usepackage[utf8]{inputenc}
%\usepackage{avant}

\usepackage[T1]{fontenc}
\usepackage{textcomp}
\usepackage[variablett]{lmodern}

\usepackage{courier}
\usepackage{listings}
\usepackage{pgf, pgffor}
\usepackage{lstlinebgrd}

% Thanks a lot to Daniel (https://tex.stackexchange.com/users/3751/daniel) for sharing his codehighlighting snippet:
% https://tex.stackexchange.com/questions/8851/how-can-i-highlight-some-lines-from-source-code
% SNIPPET_START
\makeatletter
%%%%%%%%%%%%%%%%%%%%%%%%%%%%%%%%%%%%%%%%%%%%%%%%%%%%%%%%%%%%%%%%%%%%%%%%%%%%%%
%
% \btIfInRange{number}{range list}{TRUE}{FALSE}
%
% Test in int number <number> is element of a (comma separated) list of ranges
% (such as: {1,3-5,7,10-12,14}) and processes <TRUE> or <FALSE> respectively

\newcount\bt@rangea
\newcount\bt@rangeb

\newcommand\btIfInRange[2]{%
    \global\let\bt@inrange\@secondoftwo%
    \edef\bt@rangelist{#2}%
    \foreach \range in \bt@rangelist {%
        \afterassignment\bt@getrangeb%
        \bt@rangea=0\range\relax%
        \pgfmathtruncatemacro\result{ ( #1 >= \bt@rangea) && (#1 <= \bt@rangeb) }%
        \ifnum\result=1\relax%
            \breakforeach%
            \global\let\bt@inrange\@firstoftwo%
        \fi%
    }%
    \bt@inrange%
}
\newcommand\bt@getrangeb{%
    \@ifnextchar\relax%
        {\bt@rangeb=\bt@rangea}%
        {\@getrangeb}%
}
\def\@getrangeb-#1\relax{%
    \ifx\relax#1\relax%
        \bt@rangeb=100000%   \maxdimen is too large for pgfmath
    \else%
        \bt@rangeb=#1\relax%
    \fi%
}

%%%%%%%%%%%%%%%%%%%%%%%%%%%%%%%%%%%%%%%%%%%%%%%%%%%%%%%%%%%%%%%%%%%%%%%%%%%%%%
%
% \btLstHL<overlay spec>{range list}
%
% TODO BUG: \btLstHL commands can not yet be accumulated if more than one overlay spec match.
% 
\newcommand<>{\btLstHL}[1]{%
  \only#2{\btIfInRange{\value{lstnumber}}{#1}{\color{orange!30}\def\lst@linebgrdcmd{\color@block}}{\def\lst@linebgrdcmd####1####2####3{}}}%
}%
\makeatother
% SNIPPET_END


\lstset{
 basicstyle=\ttfamily,
 columns=fullflexible,
 upquote,
 keepspaces,
 escapeinside=||,
 literate={*}{{\char42}}1
 {-}{{\char45}}1
}

\usepackage{tabulary}
\usepackage{multirow}
\usepackage{dcolumn}
\newcolumntype{.}{D{.}{.}{-1}}
\usepackage{booktabs}
\usepackage{url}

\useoutertheme[footline=authorinstitutetitle,subsection=false]{miniframes}
\usetheme{Madrid}
%\usecolortheme[RGB={52,147,54}]{structure}
\usecolortheme{seagull}
\setbeamertemplate{navigation symbols}{}
%\setbeamertemplate{blocks}[rounded]

\setbeamercolor{footlinecolor}{fg=black,bg=white}
\setbeamertemplate{footline}{%
%  \pgfputat{\pgfxy(12,6.5)}{\pgfbox[center,base]{\includegraphics[width=1.6cm]{template_images/bert_kopf.png}}}
  \begin{beamercolorbox}[sep=0.5em,wd=\paperwidth]{footlinecolor}
  \begin{columns}
    \column{11cm}
      \tiny{\insertshortauthor}
      \vspace{0.05cm}
      \hrule
      \vspace{0.08cm}
      \tiny{\insertshorttitle}
    %\column{2cm}\hfill\includegraphics[width=2cm]{template_images/Logo_FHFFM}
    \column{4cm}\hfill %\includegraphics[width=1.5cm]{template_images/Logo_FHFFM_klein}
  \end{columns}
  \end{beamercolorbox}%
}


\title[cli4clj - Easing the Implementation of Interactive Command Line Interfaces in Clojure for ``Everyone'']{cli4clj\\Easing the Implementation of Interactive Command Line Interfaces in Clojure for ``Everyone''}
%\subtitle{Event-driven Network Analysis and Surveillance\\(ENeAS)}
\author[Ruediger Gad - Terma GmbH, Space - Darmstadt, Germany - @ruedigergad]{Ruediger Gad}
\institute[]{
  Terma GmbH, Space, Darmstadt, Germany
  }
\date{:clojureD\\2019-02-23}
%\logo{\includegraphics[width=3cm]{template_images/Logo_FHFFM.pdf}}


\begin{document}

  \begin{frame}
      \titlepage
  \end{frame}

  \begin{frame}
      \frametitle{What? \& Why?}

      \begin{itemize}
          \item Interactive Command Line Interfaces (CLIs)
          \item Clojure REPL
              \begin{itemize}
                  \item Powerful \boldmath$+$
                  \item Requires Clojure Knowledge $-$?
                  \item Typical Users: Developers
              \end{itemize}
          \item cli4clj
              \begin{itemize}
                  \item ``CLIs for Everyone''
                  \item Ease Use \& Implementation
              \end{itemize}
      \end{itemize}
  \end{frame}


\begin{frame}[fragile]
\frametitle{Developer Perspective: cli4clj Configuration/Implementation Example}
\begin{lstlisting}[
    linebackgroundcolor={%
        \btLstHL<1>{2}%
        \btLstHL<2>{6-7}%
        \btLstHL<3>{8}%
        \btLstHL<4>{8-10}%
        \btLstHL<5>{11}%
        \btLstHL<6>{11-12}%
        \btLstHL<7>{3,13}%
        \btLstHL<8>{14}%
    }]
(ns cli4clj.minimal-example (:gen-class)
  (:require (cli4clj [cli :as cli])))
(defn divide [x y] (/ x y)) ;;; Used for example below.

(defn -main [& args]
  (cli/start-cli
    {:cmds
      {:test-cmd {:fn #(println "This is a test.")
                  :short-info "Test Command"
                  :long-info "Prints a test message."}
       :add {:fn (fn [summand1 summand2] (+ summand1 summand2))
             :completion-hint "Enter two values to add."}
       :divide {:fn divide}}
:allow-eval true, :alternate-scrolling (some #(= % "alt") args)}))
\end{lstlisting}
\end{frame}


\begin{frame}[fragile]
\frametitle{Application User Perspective: ``Basic Commands''}
\begin{lstlisting}[
    linebackgroundcolor={%
        \btLstHL<1>{1}%
        \btLstHL<2>{2,4,6}%
        \btLstHL<3>{2-7}%
    }]
cli# 
cli# test-cmd
This is a test.
cli# add 1 2
3
cli# divide 2 3
2/3
\end{lstlisting}
\end{frame}

\begin{frame}[fragile]
\frametitle{Application User Perspective: Help}
\begin{lstlisting}[
    linebackgroundcolor={%
        \btLstHL<1>{1}%
        \btLstHL<2>{2-8}%
    }]
cli# help
add -- Arguments: [[summand1 summand2]]
...
divide -- Arguments: [[x y]]
...
test-cmd
	Test Command
	Prints a test message.
\end{lstlisting}
\end{frame}

\begin{frame}[fragile]
    \frametitle{Application User Perspective: Tab Completion}
\begin{lstlisting}[
    linebackgroundcolor={%
        \btLstHL<1>{1-2}%
        \btLstHL<2>{3-4}%
        \btLstHL<3>{5-6}%
    }]
cli# <TAB>
...   add   ...  divide  ...  help  ...  test-cmd   
cli# a<TAB>
cli# add
cli# add <TAB>
Arguments: [[summand1 summand2]]   Enter two values to add.           
...
\end{lstlisting}
\end{frame}

\begin{frame}[fragile]
    \frametitle{Application User Perspective: Clojure Interoperability}
\begin{lstlisting}[
    linebackgroundcolor={%
        \btLstHL<1>{1,3}%
    }]
cli# (reduce add [1 7 0 1])
9
cli# (map divide [1 7 0 1] [1 8 6 4])
(1 7/8 0 1/4)
\end{lstlisting}
\end{frame}

\begin{frame}[fragile]
    \frametitle{Application User Perspective: Alternate Scrolling Mode}
\begin{lstlisting}[
    linebackgroundcolor={%
        \btLstHL<1>{7}%
        \btLstHL<2>{8}%
        \btLstHL<3>{1,3,5}%
        \btLstHL<4>{1-6}%
    }]
> test-cmd
This is a test.
> add 1 2
3
> divide 2 3
2/3
_____________________________________________
cli# 
 
 
 
 
\end{lstlisting}
\end{frame}

\begin{frame}[fragile]
    \frametitle{Developer Perspective: Automated Testing}
\begin{lstlisting}[
    linebackgroundcolor={%
        \btLstHL<1>{4-5}%
        \btLstHL<2>{8-9}%
        \btLstHL<3>{8-11}%
        \btLstHL<4>{12-14}%
    }]
(ns cli4clj.test.minimal-example
  (:require
    (clojure [test :as test])
    (cli4clj [cli-tests :as cli-tests]
             [minimal-example :as mini-example])))

(test/deftest example-test
  (let [test-cmd-input ["add 1 2"
                        "divide 3 2"]
        out-string (cli-tests/test-cli-stdout
                     #(mini-example/-main "") test-cmd-input)]
    (test/is (=
               (cli-tests/expected-string ["3" "3/2"])
               out-string))))
\end{lstlisting}
\end{frame}


%  \begin{frame}
%    \frametitle{What if throughput demands increase further?}
%      \begin{columns}
%          \column{9cm}
%    \begin{itemize}
%        \item<2-> Do nothing?\\
%              $\rightarrow$ Random Drops of ``Rows''
%        \item<3-> Apply sampling?\\
%              $\rightarrow$ More ``Controlled'' Drops of Rows
%        \item<4-> Reduce extraction operations / extracted fields?\\
%              $\rightarrow$ ``Drop columns'' in favor of rows.\\
%        \item<5-> $\rightarrow$ Adjust DSL expression rules.
%    \end{itemize}
%          \column{5cm}
%        \includegraphics<1>[width=4.5cm]{images/columns_and_rows}
%        \includegraphics<2>[width=4.5cm]{images/columns_and_rows_random_row_drops}
%        \includegraphics<3>[width=4.5cm]{images/columns_and_rows_controlled_row_drops}
%        \includegraphics<4->[width=4.5cm]{images/columns_and_rows_controlled_column_drops}
%      \end{columns}
%  \end{frame}


  \begin{frame}
      \frametitle{More}

      \begin{itemize}
          \item Persistent History
          \item Aliases (Shortcuts)
          \item Customizable
          \item ``Embedded CLIs''
      \end{itemize}
  \end{frame}


  \begin{frame}[c]
    \frametitle{End}
    \begin{center}\url{https://github.com/ruedigergad/cli4clj}\\\includegraphics[width=12cm]{images/github_badges.png}\end{center}
    \begin{center}\url{https://ruedigergad.com/category/libs/cli4clj}\end{center}
    \vspace{0.1cm}
    \begin{block}{Thank you very much for your attention!}
      \begin{center} Questions? \end{center}
    \end{block}
    \vspace{0.1cm}
    \begin{center}Ruediger Gad\\Terma GmbH, Space\\Darmstadt, Germany\end{center}
    \begin{center}ruga@terma.com\\r.c.g@gmx.de\end{center}
  \end{frame}
\end{document}

